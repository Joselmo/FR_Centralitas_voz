\documentclass[11pt]{beamer}
\usetheme{Madrid}
\usepackage[utf8]{inputenc}
\usepackage{amsmath}
\usepackage{amsfonts}
\usepackage{amssymb}
\usepackage{graphicx}
\author{Elena María Gómez Ríos y Jose Luis Martínez Ortiz}
\title{Centralitas de voz virtual} % (asterisk, 3CX, telefacil.com)
%\setbeamercovered{transparent} 
%\setbeamertemplate{navigation symbols}{} 
%\logo{} 
%\institute{} 
%\date{} 
\subject{Fundamentos de Redes} 
\begin{document}

\begin{frame}
\titlepage
\end{frame}

%\begin{frame}
%\tableofcontents
%\end{frame}

\begin{frame}{Apartados recomendables}
\begin{enumerate}
\item Introducción motivadora
\item Ejemplos de uso, variantes, tipos, etc.
\item Aplicaciones y plataformas existentes.
\item Modelo de negocio y/o explotación.
\item Prestaciones técnicas del servicio/aplicación.
\item Arquitectura software y hardware.
\item Protocolos, cabeceras, campos y funcionalidades.
\item Demostración con WireShark.
\item Uso de máquinas virtuales.
\item otros.
\end{enumerate}
\end{frame}


\begin{frame}{Introducción motivadora}
% http://www.voz.com/centralita-virtual.php
\begin{enumerate}[]
\item \begin{center}
¿Qué son las Centralitas de Voz?
\end{center}
\item 
\item \begin{center}
Es un Sistema Telefónico Avanzado por el cual se puede obtener tantas extensiones como se desee
de forma virtual.
\end{center}
\item 
\item \begin{center}
Una extensión es cada puesto de teléfono.
\end{center}

\end{enumerate}
\end{frame}


\begin{frame}{Introducción motivadora}
¿Qué ventajas tiene frente a una centralita física?\\

\begin{enumerate}[$\bullet$]
\item No se queda Obsoleta con el tiempo.
\item No necesita canales de voz internos dentro de la centralita.
% Es más se pueden crear tantos canales de voz internos como se necesite.
\item No tiene coste de establecimiento de llamada en la red interna.
\item Utiliza la red de internet.
\item Un ahorro considerable con respecto a la telefonía tradicional.
\end{enumerate}
\end{frame}

\begin{frame}{Distribuidores de Centralitas}
Algunas compañías de servicio de centralitas nacionales son:
\begin{enumerate}
\item netelip
\item VOZ.com
\item Megacall
\end{enumerate}
\vspace*{1cm}
En el ámbito internacional están:
\begin{enumerate}
\item Digium (The Asterisk Company)
\item 3CX
\item AVOXI
\end{enumerate}

\end{frame}


\begin{frame}{Ejemplos de uso}
\begin{figure}[H] %con el [H] le obligamos a situar aquí la figura
\centering
\includegraphics[scale=0.4]{./imagenes/logos.png}
\end{figure}

\end{frame}


\begin{frame}{Modelos de Negocio}
Principalmente el modelo de negocio es el de vender la centralita de voz 
ya montada y funcionando a empresas y entidades públicas por un coste 
mensual dependiendo de la amplitud de la centralita, además del hardware
necesario para su utilización como teléfonos switches etc.
\\
\vspace*{1cm}
Aunque ``Asterisk'' se centra en dar un software gratuito para que terceros vendan centralitas a partir de este. El modelo de negocio
de ``Asterisk'' es el de donaciones y patrocinadores.
\end{frame}



\begin{frame}{Protocolos VOIP}
Para llevar a cabo la transmisión en tiempo real de una comunicación 
utiliza un protocolo de tipo \textbf{VOIP} (Voice over IP), algunos más
famosos son:
\begin{enumerate}
\item Skype, propiedad de \texttt{Microsoft}
\item IAX2, original de \texttt{Asterisk} y libre
\item SCCP, propiedad de \texttt{CISCO}
\end{enumerate}
\end{frame}

\begin{frame}{Protocolo IAX2}
Utiliza UDP y está orientado al streaming media, pero fue diseñado principalmente para llamadas de voz por IP.\\
\vspace*{1cm}

Por defecto utiliza el puerto 4569 y transmite los datos por ``in-band'' 
lo que hace que el protocolo sea muy transparente para el cortafuegos y
 así ser muy rápido en redes internas.\\
\vspace*{0.5cm}
Soporta ``\texttt{Trunking red}'', lo que permite que enviar varias llamadas a la vez en un único paquete de UDP, con el ahorro de latencia que esto supone. \\
\vspace*{0.5cm}
+info en el RCF 5456
\end{frame}

\end{document}