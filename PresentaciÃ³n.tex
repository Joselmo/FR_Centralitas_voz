\documentclass[11pt]{beamer}
\usetheme{Madrid}
\usepackage[utf8]{inputenc}
\usepackage{amsmath}
\usepackage{amsfonts}
\usepackage{amssymb}
\usepackage{graphicx}
\author{Elena María Gómez Ríos y Jose Luis Martínez Ortiz}
\title{Centralitas de voz virtual} % (asterisk, 3CX, telefacil.com)
%\setbeamercovered{transparent} 
%\setbeamertemplate{navigation symbols}{} 
%\logo{} 
%\institute{} 
%\date{} 
\subject{Fundamentos de Redes} 
\begin{document}

\begin{frame}
\titlepage
\end{frame}

%\begin{frame}
%\tableofcontents
%\end{frame}

\begin{frame}{Apartados recomendables}
\begin{enumerate}
\item Introducción motivadora
\item Protocolos, cabeceras, campos y funcionalidades.
\item Ejemplos de uso, variantes, tipos, etc.
\item Aplicaciones y plataformas existentes.
\item Arquitectura software y hardware.
\item Prestaciones técnicas del servicio/aplicación.
\item Modelo de negocio y/o explotación.
\item Demostración con WireShark.
\item Uso de máquinas virtuales.
\item otros.
\end{enumerate}
\end{frame}

\end{document}